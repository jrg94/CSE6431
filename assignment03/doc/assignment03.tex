\documentclass{article}

\usepackage{fancyhdr}
\usepackage{extramarks}
\usepackage{minted}
\usepackage{color}
\usepackage[english]{babel}

%
% Basic Document Settings
%

\topmargin=-0.45in
\evensidemargin=0in
\oddsidemargin=0in
\textwidth=6.5in
\textheight=9.0in
\headsep=0.25in

\linespread{1.1}

\pagestyle{fancy}
\lhead{\hmwkAuthorName}
\chead{\hmwkClass\ (\hmwkClassInstructor): \hmwkTitle}
\rhead{\firstxmark}
\lfoot{\lastxmark}
\cfoot{\thepage}

\renewcommand\headrulewidth{0.4pt}
\renewcommand\footrulewidth{0.4pt}

\setlength\parindent{0pt}
\setlength{\parskip}{1em}

%
% Minted Settings
%

\setminted{frame=lines}
\setminted{linenos}
\setminted{autogobble}

%
% Create Problem Sections
%

\newcommand{\enterProblemHeader}[1]{
    \nobreak\extramarks{}{Problem \arabic{#1} continued on next page\ldots}\nobreak{}
    \nobreak\extramarks{Problem \arabic{#1} (continued)}{Problem \arabic{#1} continued on next page\ldots}\nobreak{}
}

\newcommand{\exitProblemHeader}[1]{
    \nobreak\extramarks{Problem \arabic{#1} (continued)}{Problem \arabic{#1} continued on next page\ldots}\nobreak{}
    \stepcounter{#1}
    \nobreak\extramarks{Problem \arabic{#1}}{}\nobreak{}
}

\setcounter{secnumdepth}{0}
\newcounter{partCounter}
\newcounter{homeworkProblemCounter}
\setcounter{homeworkProblemCounter}{1}
\nobreak\extramarks{Problem \arabic{homeworkProblemCounter}}{}\nobreak{}

%
% Homework Problem Environment
%
% This environment takes an optional argument. When given, it will adjust the
% problem counter. This is useful for when the problems given for your
% assignment aren't sequential. See the last 3 problems of this template for an
% example.
%
\newenvironment{homeworkProblem}[1][-1]{
    \ifnum#1>0
        \setcounter{homeworkProblemCounter}{#1}
    \fi
    \section{Problem \arabic{homeworkProblemCounter}}
    \setcounter{partCounter}{1}
    \enterProblemHeader{homeworkProblemCounter}
}{
    \exitProblemHeader{homeworkProblemCounter}
}

%
% Homework Details
%   - Title
%   - Due date
%   - Class
%   - Section/Time
%   - Instructor
%   - Author
%

\newcommand{\hmwkTitle}{Assignment\ \#3}
\newcommand{\hmwkDueDate}{March 28, 2019}
\newcommand{\hmwkClass}{CSE 6431}
\newcommand{\hmwkClassInstructor}{Professor Qin}
\newcommand{\hmwkAuthorName}{\textbf{Jeremy Grifski}}

%
% Title Page
%

\title{
    \vspace{2in}
    \textmd{\textbf{\hmwkClass:\ \hmwkTitle}}\\
    \normalsize\vspace{0.1in}\small{Due\ on\ \hmwkDueDate\ at 11:10am}\\
    \vspace{0.1in}\large{\textit{\hmwkClassInstructor}}
    \vspace{3in}
}

\author{\hmwkAuthorName}
\date{}

\renewcommand{\part}[1]{\textbf{\large Part \Alph{partCounter}}\stepcounter{partCounter}\\}

% Alias for the Solution section header
\newcommand{\solution}{\textbf{\large Solution}}

\begin{document}

\maketitle

\pagebreak

\begin{homeworkProblem}

    \textbf{
        In the class, we discussed a voting mechanism for reading and writing
        from replicated data. If R is the number of messages exchanged for
        reading and W is the number of messages exchanged for writing, then we
        know that R + W is O(N), where N is the number of sites on which data is
        replicated. Develop a scheme in which both read and write operations will
        require O(sqrt(N)) messages. Include arguments on why you think it will
        work correctly.
    }

\end{homeworkProblem}

\begin{homeworkProblem}

    \textbf{
        Show that when checkpoints are taken after every K (K > 1) messages are
        sent, the recovery mechanism can suffer from the domino effect. Assume
        that a process takes a checkpoint immediately after sending the Kth
        message, but before doing anything else.
    }

    As long as K > 1, the system can suffer the domino effect. Since there are
    always messages sent between checkpoints, there is always an opportunity
    where the recovery mechanism causes an orphan message. Orphan messages will
    result in another recovery which could create another orphan message.
    Naturally, the domino effect follows.

\end{homeworkProblem}

\pagebreak

\begin{homeworkProblem}

    \textbf{
        This question is about replication in distributed systems. We are given
        a coding scheme, in which a file of size F is broken into n parts of size
        F/m, such that any m of these parts are sufficient to reconstruct the
        file. Here, n ≥ m. If we have n sites in a distributed system, we can
        store each such part on each site. What potential advantages and
        disadvantages does his scheme have, as compared to the normal
        replication of files? Show how Gifford's voting algorithm needs to be
        modified. Clearly state the constraints on read and write quorum that
        are required. Give a brief argument as to how the correctness will be
        maintained.
    }

    In terms of advantages, the new scheme is quite scalable. If we want a more
    reliable system, we can increase the ratio of n to m. In other words, we
    can allow for significantly more faults because we only need m copies of a
    file to reconstruct that file. Meanwhile, we have to option to reduce the
    ratio of n to m for the sake of speed. In other words, there are less sites
    needed for data replication, so there are less read/write messages needed
    overall. In addition, since we only need to acquire m chunks of data at any
    given time, we can proceed once we receive those m chunks.

    On the flip side, it's possible that we only grab m chunks of data and
    reconstruct a corrupt file. Without proper corruption detection, there's no
    way of confirming that our file is corrupt. In addition, file reconstruction
    probably has some sort of overhead that the client has to worry about. 

\end{homeworkProblem}

\end{document}
