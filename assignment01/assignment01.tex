\documentclass{article}

\usepackage{fancyhdr}
\usepackage{extramarks}
\usepackage{amsmath}
\usepackage{amsthm}
\usepackage{amsfonts}
\usepackage{tikz}
\usepackage[plain]{algorithm}
\usepackage{algpseudocode}
\usepackage{minted}
\usepackage{color}

\usetikzlibrary{automata,positioning}

%
% Basic Document Settings
%

\topmargin=-0.45in
\evensidemargin=0in
\oddsidemargin=0in
\textwidth=6.5in
\textheight=9.0in
\headsep=0.25in

\linespread{1.1}

\pagestyle{fancy}
\lhead{\hmwkAuthorName}
\chead{\hmwkClass\ (\hmwkClassInstructor): \hmwkTitle}
\rhead{\firstxmark}
\lfoot{\lastxmark}
\cfoot{\thepage}

\renewcommand\headrulewidth{0.4pt}
\renewcommand\footrulewidth{0.4pt}

\setlength\parindent{0pt}

\definecolor{light-gray}{rgb}{0.95,0.95,0.95}
\setminted{bgcolor=light-gray}

%
% Create Problem Sections
%

\newcommand{\enterProblemHeader}[1]{
    \nobreak\extramarks{}{Problem \arabic{#1} continued on next page\ldots}\nobreak{}
    \nobreak\extramarks{Problem \arabic{#1} (continued)}{Problem \arabic{#1} continued on next page\ldots}\nobreak{}
}

\newcommand{\exitProblemHeader}[1]{
    \nobreak\extramarks{Problem \arabic{#1} (continued)}{Problem \arabic{#1} continued on next page\ldots}\nobreak{}
    \stepcounter{#1}
    \nobreak\extramarks{Problem \arabic{#1}}{}\nobreak{}
}

\setcounter{secnumdepth}{0}
\newcounter{partCounter}
\newcounter{homeworkProblemCounter}
\setcounter{homeworkProblemCounter}{1}
\nobreak\extramarks{Problem \arabic{homeworkProblemCounter}}{}\nobreak{}

%
% Homework Problem Environment
%
% This environment takes an optional argument. When given, it will adjust the
% problem counter. This is useful for when the problems given for your
% assignment aren't sequential. See the last 3 problems of this template for an
% example.
%
\newenvironment{homeworkProblem}[1][-1]{
    \ifnum#1>0
        \setcounter{homeworkProblemCounter}{#1}
    \fi
    \section{Problem \arabic{homeworkProblemCounter}}
    \setcounter{partCounter}{1}
    \enterProblemHeader{homeworkProblemCounter}
}{
    \exitProblemHeader{homeworkProblemCounter}
}

%
% Homework Details
%   - Title
%   - Due date
%   - Class
%   - Section/Time
%   - Instructor
%   - Author
%

\newcommand{\hmwkTitle}{Assignment\ \#1}
\newcommand{\hmwkDueDate}{February 05, 2019}
\newcommand{\hmwkClass}{CSE 6431}
\newcommand{\hmwkClassInstructor}{Professor Qin}
\newcommand{\hmwkAuthorName}{\textbf{Jeremy Grifski}}

%
% Title Page
%

\title{
    \vspace{2in}
    \textmd{\textbf{\hmwkClass:\ \hmwkTitle}}\\
    \normalsize\vspace{0.1in}\small{Due\ on\ \hmwkDueDate\ at 11:10am}\\
    \vspace{0.1in}\large{\textit{\hmwkClassInstructor}}
    \vspace{3in}
}

\author{\hmwkAuthorName}
\date{}

\renewcommand{\part}[1]{\textbf{\large Part \Alph{partCounter}}\stepcounter{partCounter}\\}

%
% Various Helper Commands
%

% Useful for algorithms
\newcommand{\alg}[1]{\textsc{\bfseries \footnotesize #1}}

% For derivatives
\newcommand{\deriv}[1]{\frac{\mathrm{d}}{\mathrm{d}x} (#1)}

% For partial derivatives
\newcommand{\pderiv}[2]{\frac{\partial}{\partial #1} (#2)}

% Integral dx
\newcommand{\dx}{\mathrm{d}x}

% Alias for the Solution section header
\newcommand{\solution}{\textbf{\large Solution}}

% Probability commands: Expectation, Variance, Covariance, Bias
\newcommand{\E}{\mathrm{E}}
\newcommand{\Var}{\mathrm{Var}}
\newcommand{\Cov}{\mathrm{Cov}}
\newcommand{\Bias}{\mathrm{Bias}}

\begin{document}

\maketitle

\pagebreak

\begin{homeworkProblem}

    \textbf{
        Implement a solution with writers’ priority to the readers/writers problem
        using semaphores.
    }

    \begin{minted}[gobble=8, linenos]{text}
        Procedure reader
          P(reader_mutex)
          if readers = 0 then
            readers = readers + 1
            P(writer_mutex)
          else
            readers = readers + 1
          V(reader_mutex)

          <read file>

          P(reader_mutex)
          readers = readers - 1
          if readers == 0 then V(writer_mutex)
          V(reader_mutex)

        Procedure writer
          P(sr_mutex)
          P(writer_mutex)

          <write file>

          V(writer_mutex)
          V(sr_mutex)
    \end{minted}

\end{homeworkProblem}

\pagebreak

\begin{homeworkProblem}

    \textbf{
        Implement a solution to the readers/writers problem using monitors which
        execute the requests in FCFS manner. If there are consecutive readers,
        they should be able to access the file concurrently.
    }

    \begin{minted}[gobble=8, linenos]{text}
        Procedure startRead
          begin
            readers = readers + 1;
          end

        Procedure endRead
          begin
            readers = readers - 1;
            if (readers == 0) then writer.signal;
          end

        Procedure writer
          begin
            if (readers > 0) then writer.wait;
            <write file>
          end
    \end{minted}

\end{homeworkProblem}

\begin{homeworkProblem}

    \textbf{
        Write a semaphore-based solution to the reader–writers problem that works as
        follows: If readers and writers are both waiting, then it alternates between
        readers and  writers. Otherwise, it processes them normally, i.e., readers
        concurrently and writers serially.
    }

\end{homeworkProblem}

\pagebreak

\begin{homeworkProblem}

    \textbf{
        Write a monitor-based solution to the above problem.
    }

\end{homeworkProblem}

\pagebreak

\begin{homeworkProblem}
    Prove a polynomial of degree \(k\), \(a_kn^k + a_{k - 1}n^{k - 1} + \hdots
    + a_1n^1 + a_0n^0\) is a member of \(\Theta(n^k)\) where \(a_k \hdots a_0\)
    are nonnegative constants.

    \begin{proof}
        To prove that \(a_kn^k + a_{k - 1}n^{k - 1} + \hdots + a_1n^1 +
        a_0n^0\), we must show the following:

        \[
            \exists c_1 \exists c_2 \forall n \geq n_0,\ {c_1 \cdot g(n) \leq
            f(n) \leq c_2 \cdot g(n)}
        \]

        For the first inequality, it is easy to see that it holds because no
        matter what the constants are, \(n^k \leq a_kn^k + a_{k - 1}n^{k - 1} +
        \hdots + a_1n^1 + a_0n^0\) even if \(c_1 = 1\) and \(n_0 = 1\).  This
        is because \(n^k \leq c_1 \cdot a_kn^k\) for any nonnegative constant,
        \(c_1\) and \(a_k\).
        \\

        Taking the second inequality, we prove it in the following way.
        By summation, \(\sum\limits_{i=0}^k a_i\) will give us a new constant,
        \(A\). By taking this value of \(A\), we can then do the following:

        \[
            \begin{split}
                a_kn^k + a_{k - 1}n^{k - 1} + \hdots + a_1n^1 + a_0n^0 &=
                \\
                &\leq (a_k + a_{k - 1} \hdots a_1 + a_0) \cdot n^k
                \\
                &= A \cdot n^k
                \\
                &\leq c_2 \cdot n^k
            \end{split}
        \]

        where \(n_0 = 1\) and \(c_2 = A\). \(c_2\) is just a constant. Thus the
        proof is complete.
    \end{proof}
\end{homeworkProblem}

\pagebreak

%
% Non sequential homework problems
%

% Jump to problem 18
\begin{homeworkProblem}[18]
    Evaluate \(\sum_{k=1}^{5} k^2\) and \(\sum_{k=1}^{5} (k - 1)^2\).
\end{homeworkProblem}

% Continue counting to 19
\begin{homeworkProblem}
    Find the derivative of \(f(x) = x^4 + 3x^2 - 2\)
\end{homeworkProblem}

% Go back to where we left off
\begin{homeworkProblem}[6]
    Evaluate the integrals
    \(\int_0^1 (1 - x^2) \dx\)
    and
    \(\int_1^{\infty} \frac{1}{x^2} \dx\).
\end{homeworkProblem}

\end{document}
