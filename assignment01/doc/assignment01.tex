\documentclass{article}

\usepackage{fancyhdr}
\usepackage{extramarks}
\usepackage{minted}
\usepackage{color}

%
% Basic Document Settings
%

\topmargin=-0.45in
\evensidemargin=0in
\oddsidemargin=0in
\textwidth=6.5in
\textheight=9.0in
\headsep=0.25in

\linespread{1.1}

\pagestyle{fancy}
\lhead{\hmwkAuthorName}
\chead{\hmwkClass\ (\hmwkClassInstructor): \hmwkTitle}
\rhead{\firstxmark}
\lfoot{\lastxmark}
\cfoot{\thepage}

\renewcommand\headrulewidth{0.4pt}
\renewcommand\footrulewidth{0.4pt}

\setlength\parindent{0pt}

%
% Minted Settings
%

\setminted{frame=lines}
\setminted{linenos}
\setminted{autogobble}

%
% Create Problem Sections
%

\newcommand{\enterProblemHeader}[1]{
    \nobreak\extramarks{}{Problem \arabic{#1} continued on next page\ldots}\nobreak{}
    \nobreak\extramarks{Problem \arabic{#1} (continued)}{Problem \arabic{#1} continued on next page\ldots}\nobreak{}
}

\newcommand{\exitProblemHeader}[1]{
    \nobreak\extramarks{Problem \arabic{#1} (continued)}{Problem \arabic{#1} continued on next page\ldots}\nobreak{}
    \stepcounter{#1}
    \nobreak\extramarks{Problem \arabic{#1}}{}\nobreak{}
}

\setcounter{secnumdepth}{0}
\newcounter{partCounter}
\newcounter{homeworkProblemCounter}
\setcounter{homeworkProblemCounter}{1}
\nobreak\extramarks{Problem \arabic{homeworkProblemCounter}}{}\nobreak{}

%
% Homework Problem Environment
%
% This environment takes an optional argument. When given, it will adjust the
% problem counter. This is useful for when the problems given for your
% assignment aren't sequential. See the last 3 problems of this template for an
% example.
%
\newenvironment{homeworkProblem}[1][-1]{
    \ifnum#1>0
        \setcounter{homeworkProblemCounter}{#1}
    \fi
    \section{Problem \arabic{homeworkProblemCounter}}
    \setcounter{partCounter}{1}
    \enterProblemHeader{homeworkProblemCounter}
}{
    \exitProblemHeader{homeworkProblemCounter}
}

%
% Homework Details
%   - Title
%   - Due date
%   - Class
%   - Section/Time
%   - Instructor
%   - Author
%

\newcommand{\hmwkTitle}{Assignment\ \#1}
\newcommand{\hmwkDueDate}{February 05, 2019}
\newcommand{\hmwkClass}{CSE 6431}
\newcommand{\hmwkClassInstructor}{Professor Qin}
\newcommand{\hmwkAuthorName}{\textbf{Jeremy Grifski}}

%
% Title Page
%

\title{
    \vspace{2in}
    \textmd{\textbf{\hmwkClass:\ \hmwkTitle}}\\
    \normalsize\vspace{0.1in}\small{Due\ on\ \hmwkDueDate\ at 11:10am}\\
    \vspace{0.1in}\large{\textit{\hmwkClassInstructor}}
    \vspace{3in}
}

\author{\hmwkAuthorName}
\date{}

\renewcommand{\part}[1]{\textbf{\large Part \Alph{partCounter}}\stepcounter{partCounter}\\}

%
% Various Helper Commands
%


% For derivatives
\newcommand{\deriv}[1]{\frac{\mathrm{d}}{\mathrm{d}x} (#1)}

% For partial derivatives
\newcommand{\pderiv}[2]{\frac{\partial}{\partial #1} (#2)}

% Integral dx
\newcommand{\dx}{\mathrm{d}x}

% Alias for the Solution section header
\newcommand{\solution}{\textbf{\large Solution}}

% Probability commands: Expectation, Variance, Covariance, Bias
\newcommand{\E}{\mathrm{E}}
\newcommand{\Var}{\mathrm{Var}}
\newcommand{\Cov}{\mathrm{Cov}}
\newcommand{\Bias}{\mathrm{Bias}}

\begin{document}

\maketitle

\pagebreak

\begin{homeworkProblem}

    \textbf{
        Implement a solution with writers’ priority to the readers/writers problem
        using semaphores.
    }

    \begin{minted}{text}
        Procedure reader
          P(reader_mutex)
          if readers = 0 then
            readers = readers + 1
            P(writer_mutex)
          else
            readers = readers + 1
          V(reader_mutex)

          <read file>

          P(reader_mutex)
          readers = readers - 1
          if readers == 0 then V(writer_mutex)
          V(reader_mutex)

        Procedure writer
          P(sr_mutex)
          P(writer_mutex)

          <write file>

          V(writer_mutex)
          V(sr_mutex)
    \end{minted}

\end{homeworkProblem}

\pagebreak

\begin{homeworkProblem}

    \textbf{
        Implement a solution to the readers/writers problem using monitors which
        execute the requests in FCFS manner. If there are consecutive readers,
        they should be able to access the file concurrently.
    }//

    In this solution, we track two variables: the numbers or readers and the
    number of writers. In addition, we also use a shared queue, so that call
    jobs are queued in FCFS order. Readers can only read if there are no writers
    and writers can only writer if there are no readers. However, readers can
    read concurrently as their function is split into two.

    \begin{minted}{text}
        Procedure startRead
          begin
            if (writers > 0) then all.wait;
            readers++;
          end

        Procedure endRead
          begin
            readers--;
            if (readers == 0) then all.signal;
          end

        Procedure writer
          begin
            writers++;
            if (readers > 0) then all.wait;
            <write file>
            writers--;
            all.signal
          end
    \end{minted}

\end{homeworkProblem}

\begin{homeworkProblem}

    \textbf{
        Write a semaphore-based solution to the reader–writers problem that works as
        follows: If readers and writers are both waiting, then it alternates between
        readers and  writers. Otherwise, it processes them normally, i.e., readers
        concurrently and writers serially.
    }

    \begin{minted}{text}
        Procedure reader
          P(reader_mutex)
          if readers = 0 then
            readers = readers + 1
            P(writer_mutex)
          else
            readers = readers + 1
          V(reader_mutex)

          <read file>

          P(reader_mutex)
          readers = readers - 1
          if readers == 0 then V(writer_mutex)
          V(reader_mutex)

        Procedure writer
          P(sr_mutex)
          P(writer_mutex)

          <write file>

          V(writer_mutex)
          V(sr_mutex)
    \end{minted}

\end{homeworkProblem}

\pagebreak

\begin{homeworkProblem}

    \textbf{
        Write a monitor-based solution to the above problem.
    }

    \begin{minted}{text}
        Procedure startRead
          begin
            readers = readers + 1;
          end

        Procedure endRead
          begin
            readers = readers - 1;
            if (readers == 0) then writer.signal;
          end

        Procedure writer
          begin
            if (readers > 0) then writer.wait;
            <write file>
          end
    \end{minted}

\end{homeworkProblem}

\pagebreak

\begin{homeworkProblem}

    \textbf{
        A file is to be shared among different processes, each of which has a unique
        number. The file can be accessed simultaneously by several processes, subject
        to the following constraint: The sum of all unique numbers associated with all
        the processes concurrently accessing the file must be less than n. Write a
        monitor to coordinate accesses to the file.
    }

\end{homeworkProblem}

\pagebreak

\begin{homeworkProblem}

    \textbf{
        Using Java support for multithreading (Synchronized, wait, and notifyall),
        write a solution to the producer-consumer problem with a buffer of length N.
        Submit your solution on paper (i.e. do not worry about exact syntax or debugging).
    }\\

    The following solution demonstrates just the shared memory portion of the
    producers/consumers problem. In addition to this class, you'd need to create
    producer and consumer classes which call the produce and consume methods
    in some way.

    \begin{minted}{java}
        /**
         * A shared memory space class with functions for consumption and production.
         *
         * @author Jeremy Grifski
         */
        public class Buffer {
          private ArrayList<Integer> buffer;
          private maxSize;

          public Buffer(int maxSize) {
            this.buffer = new ArrayList<Integer>();
            this.maxSize = maxSize;
          }

          /**
           * A production method which generates a random value between 0 and 1000
           * and places it at the end of shared memory.
           */
          public synchronized void produce() {
            while (buffer.size() == maxSize) {
              try {
                wait();
              } catch (InterruptedException e) {
                System.out.println("Producer awoken!");
              }
            }
            this.buffer.add((int)(Math.random() * 1000));
            notifyAll();
          }

          /**
           * A consumption method which consumes the first value from the buffer.
           */
          public synchronized int consume() {
            while(buffer.isEmpty()) {
              try {
                wait();
              } catch (InterruptedException e) {
                System.out.println("Consumer awoken!");
              }
            }
            int value = this.buffer.remove(0);
            notifyAll();
            return value;
          }
        }
    \end{minted}

    As you can see, this solution leverages all that is asked via synchronized,
    wait, and notifyAll. The following code snippets are the class stubs for
    the producer and consumer classes:

    \begin{minted}{java}
        public class Producer extends Thread {
          private Buffer sharedMemory;

          public Producer(Buffer sharedMemory) {}

          public void run() {
            // Implement production loop
          }
        }
    \end{minted}

    \begin{minted}{java}
        public class Consumer extends Thread {
          private Buffer sharedMemory;

          public Consumer(Buffer sharedMemory) {}

          public void run() {
            // Implement consumption loop
          }
        }
    \end{minted}

\end{homeworkProblem}

\begin{homeworkProblem}

    \textbf{
        Using Java support for multithreading (Synchronized, wait, and notifyall),
        write a solution to the readers-writers problem, with exclusive writer access,
        concurrent reader access, and reader’s priority. Submit your solution on paper
        (i.e. do not worry about exact syntax or debugging).
    }

\end{homeworkProblem}

\end{document}
